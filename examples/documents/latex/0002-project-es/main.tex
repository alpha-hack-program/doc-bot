\documentclass[12pt]{article}
\usepackage[utf8]{inputenc}
\usepackage{geometry}
\geometry{a4paper, margin=1in}
\usepackage{graphicx}
\usepackage{hyperref}
\usepackage[spanish]{babel}
\usepackage{pgfplots}
\usepackage{array}
\pgfplotsset{compat=1.18} 

\title{Memorándum Extendido del Proyecto 0002}
\author{Gerente de Proyecto: Juan Pérez}
\date{24 de septiembre de 2024}

\begin{document}

\maketitle

\tableofcontents
\newpage

\section{Resumen Ejecutivo}
Este memorándum ofrece una visión completa del proyecto ABC. Cubre el contexto, el progreso, los riesgos y desafíos. Además, se presenta un análisis detallado del cronograma y del presupuesto. También se proporcionará una comparación del plan inicial con el estado actual, destacando desviaciones y estrategias de mitigación de riesgos.

\section{Contexto}
El proyecto ABC se inició para abordar las ineficiencias del sistema existente. El proyecto busca mejorar el rendimiento del sistema en un 30\%, reducir el tiempo de inactividad y mejorar la experiencia del usuario. El proyecto comenzó oficialmente el 1 de julio de 2024, con una fecha de finalización esperada en diciembre de 2024.

El proyecto involucra varios equipos, incluidos desarrollo de software, aseguramiento de la calidad y gestión de proyectos. Los principales objetivos son:
\begin{itemize}
    \item Mejorar el rendimiento del sistema en un 30\%.
    \item Mejorar la experiencia del usuario mediante una interfaz rediseñada.
    \item Reducir el tiempo de inactividad del sistema en un 50\%.
    \item Asegurar la sincronización de datos entre plataformas.
\end{itemize}

\section{Progreso Hasta la Fecha}
El proyecto se encuentra actualmente en la segunda fase, con un 60\% del trabajo de desarrollo completado. A continuación, se presenta un desglose del trabajo realizado hasta la fecha:

\begin{itemize}
    \item \textbf{Fase 1: Recolección de Requisitos} (Completado el 15 de agosto de 2024)
    \item \textbf{Fase 2: Desarrollo} (60\% completado, finalización prevista el 15 de octubre de 2024)
    \item \textbf{Fase 3: Pruebas e Implementación} (Programado para noviembre - diciembre de 2024)
\end{itemize}

\section{Cronograma del Proyecto y Hitos}
La siguiente tabla proporciona un cronograma detallado de los hitos del proyecto y el porcentaje completado:

\begin{center}
\begin{table}[h]
\small
% \fontsize{10pt}{10pt}\selectfont
\begin{tabular}{|m{4cm}|m{4cm}|m{4cm}|m{2cm}|}
\hline
\textbf{Hito} & \textbf{Fecha Planificada} & \textbf{Fecha Real} & \textbf{Estado} \\
\hline
Recolección Requisitos & 15 de agosto de 2024 & 15 de agosto de 2024 & 100\% \\
\hline
Autenticación Usuarios & 1 de septiembre de 2024 & 5 de septiembre de 2024 & 100\% \\
\hline
Integración API & 1 de octubre de 2024 & En curso & 50\% \\
\hline
Fase de Pruebas & 15 de noviembre de 2024 & Por determinar & No Iniciado \\
\hline
Implementación Final & 15 de diciembre de 2024 & Por determinar & No Iniciado \\
\hline
\end{tabular}
\end{table}
\end{center}

\section{Análisis Presupuestario}
El presupuesto del proyecto se ha mantenido dentro de los límites, aunque el retraso en la integración de API podría afectar los costos. A continuación se presenta un desglose del presupuesto asignado frente al presupuesto real para cada fase.

\begin{itemize}
    \item \textbf{Presupuesto Total:} \$500,000
    \item \textbf{Cantidad Gastada:} \$300,000 (Hasta el 24 de septiembre de 2024)
\end{itemize}

\subsection*{Desglose Detallado del Presupuesto}
\begin{itemize}
    \item Recolección de Requisitos: \$100,000 (Gastado: \$95,000)
    \item Desarrollo: \$250,000 (Gastado: \$150,000)
    \item Pruebas: \$100,000 (Asignado)
    \item Implementación: \$50,000 (Asignado)
\end{itemize}

\section{Estado Actual}
La fase de desarrollo está en curso, centrada en la integración de API. A continuación se muestra un gráfico de Gantt que ofrece una representación visual del cronograma del proyecto.

\begin{center}
\begin{tikzpicture}
    \begin{axis}[
        title={Cronograma del Proyecto (Gráfico de Gantt)},
        xlabel={Fases del Proyecto},
        ylabel={Cronograma (Meses)},
        symbolic x coords={Julio, Agosto, Septiembre, Octubre, Noviembre, Diciembre},
        xtick=data,
        ybar=0.7,
        ymin=0, ymax=6,
        bar width=15pt,
        enlarge x limits={abs=0.5cm},
        legend pos=north west
    ]
    \addplot coordinates {(Julio, 1) (Agosto, 2) (Septiembre, 2) (Octubre, 3) (Noviembre, 4) (Diciembre, 5)};
    \addlegendentry{Progreso Esperado}
    
    \addplot coordinates {(Julio, 1) (Agosto, 2) (Septiembre, 2) (Octubre, 2.5) (Noviembre, 0) (Diciembre, 0)};
    \addlegendentry{Progreso Real}
    \end{axis}
\end{tikzpicture}
\end{center}

\section{Riesgos y Mitigación}
El proyecto ha encontrado varios riesgos que podrían afectar su cronograma y presupuesto:
\begin{itemize}
    \item \textbf{Riesgo:} Retrasos en la integración de API.
    \item \textbf{Mitigación:} Se han asignado recursos adicionales para centrarse en el desarrollo de API.
    \item \textbf{Riesgo:} Potenciales retrasos en las pruebas de aceptación debido a la finalización tardía del desarrollo.
    \item \textbf{Mitigación:} Los cronogramas de prueba se ajustarán para correr en paralelo con el trabajo de desarrollo restante.
\end{itemize}

\section{Próximos Pasos}
Las siguientes tareas son prioritarias para la próxima fase del proyecto:
\begin{itemize}
    \item Completar la integración de API antes del 1 de octubre de 2024.
    \item Iniciar pruebas antes del 1 de noviembre de 2024.
    \item Asegurar la implementación del sistema para diciembre de 2024.
\end{itemize}

\section{Conclusión}
En conclusión, el proyecto ABC está progresando a pesar de algunos retrasos menores en el desarrollo. Confiamos en que el proyecto cumplirá con el cronograma revisado, y se espera que la integración de API se complete pronto. Se han asignado recursos adicionales para garantizar que las fases de pruebas e implementación se mantengan en el cronograma.

\end{document}
