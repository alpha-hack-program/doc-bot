\documentclass[12pt]{article}
\usepackage[utf8]{inputenc}
\usepackage{geometry}
\geometry{a4paper, margin=1in}
\usepackage{graphicx}
\usepackage{hyperref}
\usepackage[brazil]{babel}
\usepackage{pgfplots}
\usepackage{array}
\pgfplotsset{compat=1.18}

\title{Relatório Executivo do Projeto 0003}
\author{Gerente do Projeto: Ana Silva}
\date{24 de setembro de 2024}

\begin{document}

\maketitle

\tableofcontents
\newpage

\section*{Resumo Executivo}
\addcontentsline{toc}{section}{Resumo Executivo}
Este relatório detalha o progresso e a situação atual do projeto XYZ. Inclui uma análise do cronograma, do orçamento e dos principais riscos enfrentados. Adicionalmente, são apresentadas estratégias para mitigar atrasos e melhorar a eficiência da equipe. O projeto, iniciado em junho de 2024, visa a conclusão até janeiro de 2025.

\section*{Contexto}
\addcontentsline{toc}{section}{Contexto}
O projeto XYZ foi criado para modernizar a infraestrutura de TI da empresa, com foco na implementação de um novo sistema de gerenciamento de dados. O principal objetivo é reduzir o tempo de processamento de informações em 40\% e aumentar a segurança dos dados. A equipe do projeto inclui especialistas em TI, engenheiros de software e consultores externos.

Os principais objetivos incluem:
\begin{itemize}
    \item Reduzir o tempo de processamento de dados em 40\%.
    \item Implementar um sistema de backup automatizado.
    \item Aumentar a segurança com criptografia avançada.
    \item Melhorar a acessibilidade dos dados para diferentes departamentos.
\end{itemize}

\section*{Progresso Até o Momento}
\addcontentsline{toc}{section}{Progresso Até o Momento}
O projeto encontra-se na fase de desenvolvimento, com 70\% das tarefas concluídas. A seguir, um resumo das etapas finalizadas:

\begin{itemize}
    \item \textbf{Fase 1: Planejamento e Design} (Completado em 30 de julho de 2024)
    \item \textbf{Fase 2: Desenvolvimento} (70\% completo, conclusão prevista para 20 de outubro de 2024)
    \item \textbf{Fase 3: Testes e Implantação} (Agendado para novembro-dezembro de 2024)
\end{itemize}

\section*{Cronograma do Projeto e Marcos}
\addcontentsline{toc}{section}{Cronograma do Projeto e Marcos}
A tabela a seguir apresenta o cronograma detalhado dos marcos do projeto e o status de conclusão de cada um:

\begin{center}
\begin{tabular}{|m{4cm}|m{3cm}|m{3cm}|m{3cm}|}
\hline
\textbf{Marco} & \textbf{Data Planejada} & \textbf{Data Real} & \textbf{Status de Conclusão} \\
\hline
Planejamento Completo & 30 de julho de 2024 & 30 de julho de 2024 & 100\% \\
\hline
Desenvolvimento de API & 15 de setembro de 2024 & 20 de setembro de 2024 & 100\% \\
\hline
Integração de Banco de Dados & 1 de outubro de 2024 & Em andamento & 60\% \\
\hline
Testes de Sistema & 15 de novembro de 2024 & A definir & Não Iniciado \\
\hline
Implantação Final & 15 de janeiro de 2025 & A definir & Não Iniciado \\
\hline
\end{tabular}
\end{center}

\section*{Análise Orçamentária}
\addcontentsline{toc}{section}{Análise Orçamentária}
O orçamento do projeto está dentro dos limites, embora o atraso na integração do banco de dados possa resultar em custos adicionais. A seguir, o orçamento alocado em comparação com os valores reais até o momento:

\begin{itemize}
    \item \textbf{Orçamento Total:} R\$600.000
    \item \textbf{Total Gasto:} R\$350.000 (Até 24 de setembro de 2024)
\end{itemize}

\subsection*{Detalhamento do Orçamento}
\begin{itemize}
    \item Planejamento e Design: R\$120.000 (Gasto: R\$110.000)
    \item Desenvolvimento: R\$300.000 (Gasto: R\$180.000)
    \item Testes: R\$120.000 (Alocado)
    \item Implantação: R\$60.000 (Alocado)
\end{itemize}

\section*{Status Atual}
\addcontentsline{toc}{section}{Status Atual}
Atualmente, o projeto está focado na integração do banco de dados. A seguir, um gráfico de Gantt que ilustra o cronograma do projeto:

\begin{center}
\begin{tikzpicture}
    \begin{axis}[
        title={Cronograma do Projeto (Gráfico de Gantt)},
        xlabel={Fases do Projeto},
        ylabel={Cronograma (Meses)},
        symbolic x coords={Junho, Julho, Agosto, Setembro, Outubro, Novembro, Dezembro, Janeiro},
        xtick=data,
        ybar=0.7,
        ymin=0, ymax=8,
        bar width=15pt,
        enlarge x limits={abs=0.5cm},
        legend pos=north west
    ]
    \addplot coordinates {(Junho, 1) (Julho, 2) (Agosto, 2) (Setembro, 3) (Outubro, 4) (Novembro, 5) (Dezembro, 6) (Janeiro, 7)};
    \addlegendentry{Progresso Esperado}
    
    \addplot coordinates {(Junho, 1) (Julho, 2) (Agosto, 2) (Setembro, 3) (Outubro, 3.5) (Novembro, 0) (Dezembro, 0) (Janeiro, 0)};
    \addlegendentry{Progresso Real}
    \end{axis}
\end{tikzpicture}
\end{center}

\section*{Riscos e Mitigação}
\addcontentsline{toc}{section}{Riscos e Mitigação}
Alguns riscos identificados que podem impactar o cronograma e o orçamento do projeto incluem:
\begin{itemize}
    \item \textbf{Risco:} Atrasos na integração do banco de dados.
    \item \textbf{Mitigação:} Alocação de uma equipe dedicada para acelerar o processo de integração.
    \item \textbf{Risco:} Problemas de segurança durante os testes.
    \item \textbf{Mitigação:} Implementação de auditorias de segurança contínuas durante o desenvolvimento.
\end{itemize}

\section*{Próximos Passos}
\addcontentsline{toc}{section}{Próximos Passos}
As principais atividades para as próximas semanas são:
\begin{itemize}
    \item Finalizar a integração do banco de dados até 15 de outubro de 2024.
    \item Iniciar os testes de sistema até 1 de novembro de 2024.
    \item Garantir a conclusão da implantação até janeiro de 2025.
\end{itemize}

\section*{Conclusão}
\addcontentsline{toc}{section}{Conclusão}
O projeto XYZ está progredindo conforme o cronograma ajustado. Embora tenha havido atrasos no desenvolvimento, os recursos adicionais alocados devem garantir que o projeto seja concluído dentro do prazo revisado. Espera-se que a integração do banco de dados seja finalizada em breve, o que permitirá a execução dos testes de sistema e a implantação dentro do prazo.

\end{document}
