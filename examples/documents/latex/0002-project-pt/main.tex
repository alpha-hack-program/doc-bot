\documentclass[12pt]{article}
\usepackage[utf8]{inputenc}
\usepackage{geometry}
\geometry{a4paper, margin=1in}
\usepackage{graphicx}
\usepackage{hyperref}
\usepackage[portuguese]{babel}
\usepackage{pgfplots}
\usepackage{array}
\pgfplotsset{compat=1.18} 

\title{Memorando Estendido do Projeto 0002}
\author{Gerente de Projeto: Juan Pérez}
\date{24 de setembro de 2024}

\begin{document}

\maketitle

\tableofcontents
\newpage

\section{Resumo Executivo}
Este memorando oferece uma visão completa do projeto ABC. Ele cobre o contexto, progresso, riscos e desafios. Além disso, é apresentada uma análise detalhada do cronograma e do orçamento. Também será fornecida uma comparação entre o plano inicial e o estado atual, destacando desvios e estratégias de mitigação de riscos.

\section{Contexto}
O projeto ABC foi iniciado para resolver ineficiências no sistema existente. O projeto busca melhorar o desempenho do sistema em 30\%, reduzir o tempo de inatividade e aprimorar a experiência do usuário. O projeto foi oficialmente iniciado em 1º de julho de 2024, com data de conclusão esperada para dezembro de 2024.

O projeto envolve várias equipes, incluindo desenvolvimento de software, garantia de qualidade e gerenciamento de projetos. Os principais objetivos são:
\begin{itemize}
    \item Melhorar o desempenho do sistema em 30\%.
    \item Melhorar a experiência do usuário por meio de uma interface redesenhada.
    \item Reduzir o tempo de inatividade do sistema em 50\%.
    \item Garantir a sincronização de dados entre plataformas.
\end{itemize}

\section{Progresso até o Momento}
Atualmente, o projeto está na segunda fase, com 60\% do trabalho de desenvolvimento concluído. Abaixo está uma divisão do trabalho realizado até o momento:

\begin{itemize}
    \item \textbf{Fase 1: Coleta de Requisitos} (Concluída em 15 de agosto de 2024)
    \item \textbf{Fase 2: Desenvolvimento} (60\% concluído, conclusão prevista para 15 de outubro de 2024)
    \item \textbf{Fase 3: Testes e Implementação} (Programada para novembro - dezembro de 2024)
\end{itemize}

\section{Cronograma do Projeto e Marcos}
A tabela a seguir fornece um cronograma detalhado dos marcos do projeto e o percentual de conclusão:

\begin{center}
\begin{table}[h]
\small
% \fontsize{10pt}{10pt}\selectfont
\begin{tabular}{|m{4cm}|m{4cm}|m{4cm}|m{2cm}|}
\hline
\textbf{Marco} & \textbf{Data Planejada} & \textbf{Data Real} & \textbf{Status} \\
\hline
Coleta de Requisitos & 15 de agosto de 2024 & 15 de agosto de 2024 & 100\% \\
\hline
Autenticação de Usuários & 1º de setembro de 2024 & 5 de setembro de 2024 & 100\% \\
\hline
Integração de API & 1º de outubro de 2024 & Em andamento & 50\% \\
\hline
Fase de Testes & 15 de novembro de 2024 & A ser determinado & Não Iniciado \\
\hline
Implementação Final & 15 de dezembro de 2024 & A ser determinado & Não Iniciado \\
\hline
\end{tabular}
\end{table}
\end{center}

\section{Análise Orçamentária}
O orçamento do projeto permaneceu dentro dos limites, embora o atraso na integração da API possa impactar os custos. A seguir, uma divisão do orçamento alocado versus o orçamento real para cada fase.

\begin{itemize}
    \item \textbf{Orçamento Total:} \$500,000
    \item \textbf{Valor Gasto:} \$300,000 (Até 24 de setembro de 2024)
\end{itemize}

\subsection*{Detalhamento do Orçamento}
\begin{itemize}
    \item Coleta de Requisitos: \$100,000 (Gasto: \$95,000)
    \item Desenvolvimento: \$250,000 (Gasto: \$150,000)
    \item Testes: \$100,000 (Alocado)
    \item Implementação: \$50,000 (Alocado)
\end{itemize}

\section{Status Atual}
A fase de desenvolvimento está em andamento, focada na integração da API. Abaixo está um gráfico de Gantt que oferece uma representação visual do cronograma do projeto.

\begin{center}
\begin{tikzpicture}
    \begin{axis}[
        title={Cronograma do Projeto (Gráfico de Gantt)},
        xlabel={Fases do Projeto},
        ylabel={Cronograma (Meses)},
        symbolic x coords={Julho, Agosto, Setembro, Outubro, Novembro, Dezembro},
        xtick=data,
        ybar=0.7,
        ymin=0, ymax=6,
        bar width=15pt,
        enlarge x limits={abs=0.5cm},
        legend pos=north west
    ]
    \addplot coordinates {(Julho, 1) (Agosto, 2) (Setembro, 2) (Outubro, 3) (Novembro, 4) (Dezembro, 5)};
    \addlegendentry{Progresso Esperado}
    
    \addplot coordinates {(Julho, 1) (Agosto, 2) (Setembro, 2) (Outubro, 2.5) (Novembro, 0) (Dezembro, 0)};
    \addlegendentry{Progresso Real}
    \end{axis}
\end{tikzpicture}
\end{center}

\section{Riscos e Mitigação}
O projeto encontrou diversos riscos que podem afetar o cronograma e o orçamento:
\begin{itemize}
    \item \textbf{Risco:} Atrasos na integração da API.
    \item \textbf{Mitigação:} Recursos adicionais foram alocados para focar no desenvolvimento da API.
    \item \textbf{Risco:} Potenciais atrasos nos testes de aceitação devido à conclusão tardia do desenvolvimento.
    \item \textbf{Mitigação:} Os cronogramas de teste serão ajustados para ocorrer em paralelo com o trabalho de desenvolvimento restante.
\end{itemize}

\section{Próximos Passos}
As seguintes tarefas são prioritárias para a próxima fase do projeto:
\begin{itemize}
    \item Concluir a integração da API até 1º de outubro de 2024.
    \item Iniciar os testes antes de 1º de novembro de 2024.
    \item Garantir a implementação do sistema até dezembro de 2024.
\end{itemize}

\section{Conclusão}
Em conclusão, o projeto ABC está progredindo apesar de alguns pequenos atrasos no desenvolvimento. Estamos confiantes de que o projeto cumprirá o cronograma revisado, e a integração da API deverá ser concluída em breve. Recursos adicionais foram alocados para garantir que as fases de testes e implementação permaneçam dentro do cronograma.

\end{document}
